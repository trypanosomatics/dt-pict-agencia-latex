% -------------------------------------------------------------------
\section{Cronograma de trabajo} 
% -------------------------------------------------------------------
% max 1 pag Se presentará una tabla de doble entrada con las tareas
% desagregadas y los tiempos estimados que consumirán.
% -------------------------------------------------------------------
\makeatletter
\setlength{\@fptop}{0pt}
\makeatother

\begin{figure}[H]
	% --------------------------------------------------------------------------- %
% CRONOGRAMA as Gannt Chart
% --------------------------------------------------------------------------- %
\setganttlinklabel{s-s}{START-TO-START}
\setganttlinklabel{f-s}{FINISH-TO-START}
\setganttlinklabel{f-f}{FINISH-TO-FINISH}
\sffamily

% scale down the whole chart to fit the page 
\scalebox{0.85}{
\begin{ganttchart}[
    canvas/.append style={fill=none, draw=black!5, line width=.75pt},
    hgrid style/.style={draw=black!5, line width=.75pt},
    vgrid={*1{draw=black!5, line width=.75pt}},
    today=2,
    today rule/.style={
      draw=black!64,
      dash pattern=on 3.5pt off 4.5pt,
      line width=1.5pt
    },
    today label font=\small\bfseries,
    today label=INICIO PICT,
    title/.style={draw=none, fill=none},
    title label font=\bfseries\footnotesize,
    title label node/.append style={below=7pt},
    include title in canvas=false,
    bar label font=\mdseries\small\color{black!70},
    bar label node/.append style={left=2cm},
    bar/.append style={draw=none, fill=black!63},
    bar incomplete/.append style={fill=barblue},
    bar progress label font=\mdseries\footnotesize\color{black!70},
    group incomplete/.append style={fill=groupblue},
    group left shift=0,
    group right shift=0,
    group height=.5,
    group peaks tip position=0,
    group label node/.append style={left=.2cm},
    group progress label font=\bfseries\small,
    link/.style={-latex, line width=1.5pt, linkred},
    link label font=\scriptsize\bfseries,
    link label node/.append style={below left=-2pt and 0pt}
  ]{1}{10}
  \gantttitle[
    title label node/.append style={below left=7pt and -3pt}
  ]{SEMESTRES:\quad1}{1}
  \gantttitlelist{2,...,10}{1} \\
  \ganttgroup[progress=35]{Objetivo 1}{1}{6} \\
  \ganttbar[progress=60]{\textbf{\ref{sec:aim1}} Aim 1}{1}{4} \\
  \ganttbar[progress=0]{\textbf{\ref{sec:aim2}} Aim 2}{2}{4} \\
  \ganttbar[progress=0]{\textbf{\ref{sec:aim3}} Aim 3}{4}{5} \\
  \ganttbar[progress=0]{Publish results}{3}{6} \\[grid]
  \ganttgroup[progress=20]{Objetivo 2}{1}{10} \\
  \ganttbar[progress=40]{\textbf{\ref{sec:aim4}} Aim 4}{1}{5} \\
  \ganttbar[progress=20]{\textbf{\ref{sec:aim5}} Aim 5}{2}{7} \\
  \ganttbar[progress=0]{\textbf{\ref{sec:aim6}}  Aim 6}{5}{10}\\
  \ganttbar[progress=0]{Publish results}{7}{10}
  %\ganttlink[link type=s-s]{WBS1A}{WBS1B}
  %\ganttlink[link type=f-s]{WBS1B}{WBS1C}
  %\ganttlink[
  %  link type=f-f,
  %  link label node/.append style=left
  %]{WBS1C}{WBS1D}
\end{ganttchart}
}


	\caption{{\bfseries{Cronograma de Objetivos y Tareas}}. Este proyecto es
	continuación del proyecto PICT-0000-0000 (Agencia I+D+i). En el
	cronograma se muestran tareas ya iniciadas que forman parte de estos
	proyectos. El diagrama muestra las tareas desagregadas y los tiempos que
	consumirán cada una. La fecha de inicio de este subsidio se marca como
	referencia tentativa (línea punteada). Los semestres 1 y 2 corresponden al
	período anterior al inicio de este proyecto solicitado. Los objetivos y
	tareas completados se muestran en distintas escalas de negro/gris, aquellos
	en progreso en azul.}\label{fig:cronograma}

\end{figure}


% --------------------------------------------------------------------------- %
% Alternativa 
% --------------------------------------------------------------------------- %
% tomar un screenshot de un cronograma, ej hecho en una planilla
% e incrustarlo acá
%\begin{figure}[!h]
%  \includegraphics{figures/cronograma}
%  \caption{Cronograma.}\label{fig:cronograma}
%\end{figure}

