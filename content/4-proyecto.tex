% --------------------------------------------------------------------------- %
% Proyecto - Diseño experimental y metodos 
% --------------------------------------------------------------------------- %
% max 9 pags
% Se deberá organizar el estudio propuesto en secciones mayores
% correspondientes a los objetivos específicos y secciones menores
% correspondientes a experimentos específicos para explicar:
%  1. La base racional de cada experimento o estudio propuesto.
%  2. Como se llevara a cabo el experimento o estudio.
%  3. Que controles se usaran - en caso de ser necesarios - y porque.
%  4. Que técnicas especificas se utilizaran discutiendo aspectos
%     más críticos o modificaciones de manipulaciones habituales: 
%     Respecto a las técnicas y tecnologías empleadas (los métodos) 
%     si son parte del patrimonio del grupo y han sido descriptas en
%     publicaciones propias o en los datos preliminares - no deberán 
%     detallarse y solo deberá citarse la fuente. Explicar si se 
%     recibirá apoyo técnico de colaboradores.
%  5. Como se interpretaran los datos a la luz de lo que se quiere
%     estudiar y como se contrastará con la hipótesis de trabajo.
%  6. Tratar de evaluar los potenciales problemas y limitaciones de
%     la metodología y técnicas propuestas y en lo posible proponer
%     alternativas.
% --------------------------------------------------------------------------- %
\section{Diseño experimental y métodos}\label{sec:experimental} 

\lipsum*[2-3] 

\subsection{Aim 1}\label{sec:aim1}

Let's write some math: $(X+Y)^2=X^2+2XY+Y^2$\\

\noindent Let $f\colon X\to Y$ be a map.\\

\noindent $f\colon \mathbb{R^+}\to\mathbb{R^+}, f(x) = x^2$ is injective.\\


\subsection{Aim 2}\label{sec:aim2}

\lipsum*[5]

\begin{figure}[H]
    \centering
    \includegraphics[width=0.5\textwidth]{figuras/image-single-col}
    \caption{{\bf{Caption title.}} Caption text.}\label{fig:sample}
\end{figure}

\subsection{Aim 3}\label{sec:aim3}

\lipsum*[6]

\subsection{Aim 4}\label{sec:aim4}

\lipsum*[7]

\subsection{Aim 5}\label{sec:aim5}

\lipsum*[8]

\subsection{Aim 6}\label{sec:aim6}

\lipsum*[9]