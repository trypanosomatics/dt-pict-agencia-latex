% --------------------------------------------------------------------------- %
% Fonts
% --------------------------------------------------------------------------- %
% if using commercial fonts, place all fonts into their own './fonts' folder 
% .gitignore is already set to avoid uploading fonts to public repos

% --------------------------------------------------------------------------- %
% Option 0: No selection of fonts whatsoever
% --------------------------------------------------------------------------- %
% You will get Computer Modern fonts. 
% --------------------------------------------------------------------------- %

% --------------------------------------------------------------------------- %
% OPTION 1: Stix + Liberation Sans + Inconsolata
% --------------------------------------------------------------------------- %
% https://www.lfe.pt/latex/fonts/typography/2022/11/21/latex-fonts-part1.html
% https://github.com/linoferreira/latex-font-survey/blob/master/stix/lualatex/stix-two.tex
% \usepackage[cm-default]{fontspec}
% \usepackage{microtype}
% \usepackage{mathtools,amssymb}
% \usepackage{unicode-math}
% \usepackage{inconsolata}

% % Serif font: STIX Two (https://www.stixfonts.org)
% \setmainfont[Numbers=OldStyle]{STIX Two Text}
% % Math font: STIX Two Math
% \setmathfont{STIX Two Math}
% % Sans-serif font: Libertinus Sans
% \setsansfont[Numbers=OldStyle]{Libertinus Sans}
% % Monospaced font: Inconsolata (https://www.ctan.org/pkg/inconsolata)
% \setmonofont{inconsolata}
% END OPTION 1
% --------------------------------------------------------------------------- %


% --------------------------------------------------------------------------- %
% OPTION 2: TeX Gyre Pagella + Source Sans Pro + TeX Gyre Cursor
% --------------------------------------------------------------------------- %
% https://www.lfe.pt/latex/fonts/typography/2022/11/21/latex-fonts-part1.html
% https://tug.org/FontCatalogue/texgyrepagella
% https://tug.org/FontCatalogue/texgyrecursor/
%\usepackage{fontspec}
%\usepackage{amsfonts}
% Serif Font: TeX Gyre Pagella
%\usepackage{tgpagella}
%\setmainfont{TeX Gyre Pagella}
% Sans Serif Font: Source Sans Pro
%\usepackage{sourcesanspro}
%\setsansfont{Source Sans Pro}
% Math Font:
% Monospaced Font: 
%\usepackage{tgcursor}
%\setmonofont{TeX Gyre Cursor}
% --------------------------------------------------------------------------- %



% --------------------------------------------------------------------------- %
% OPTION 3: EB Garamond + Source Sans Pro + Garamond Math + New CM Mono
% --------------------------------------------------------------------------- %
% https://www.lfe.pt/latex/fonts/typography/2022/11/21/latex-fonts-part1.html
 \usepackage{fontspec}
% % Serif Font: 
\setmainfont[Numbers=OldStyle]{EB Garamond} 

% % Sans Serif Font: Source Sans Pro
\usepackage{sourcesanspro}
\setsansfont{Source Sans Pro}

% % Math Font
\usepackage[math-style=ISO, bold-style=ISO]{unicode-math}
\setmathfont{Garamond-Math.otf}[StylisticSet={6,7,9}]

% % Monospaced Font
\usepackage{unicode-math}
\setmonofont[ItalicFont=NewCMMono10-Italic.otf,%
    BoldFont=NewCMMono10-Bold.otf,%
    BoldItalicFont=NewCMMono10-BoldOblique.otf,%
    SmallCapsFeatures={Numbers=OldStyle}]{NewCMMono10-Regular.otf}
% --------------------------------------------------------------------------- %


% --------------------------------------------------------------------------- %
% OPTION 4: Libertinus + Inconsolata
% --------------------------------------------------------------------------- %
% https://www.lfe.pt/latex/fonts/typography/2022/11/21/latex-fonts-part1.html
%https://github.com/linoferreira/latex-font-survey/blob/master/libertine/lualatex/libertinus.tex
% \usepackage{fontspec}
% \usepackage{microtype}
% \usepackage{mathtools,amssymb}
% \usepackage{unicode-math}
% \usepackage{xcolor}

% % Serif font: Libertinus Serif (https://github.com/alif-type/libertinus)
% \setmainfont[Numbers=OldStyle]{Libertinus Serif}
% % Math font: Libertinus Math
% \setmathfont[AutoFakeBold]{Libertinus Math}
% % Sans-serif font: Libertinus Sans
% \setsansfont[Numbers=OldStyle]{Libertinus Sans}
% % Monospaced font: Inconsolata (https://www.ctan.org/pkg/inconsolata)
% \fontspec{inconsolata}
% % Display font: Libertinus Serif Display
% \newfontfamily\dispfont{Libertinus Serif Display}
% --------------------------------------------------------------------------- %


% --------------------------------------------------------------------------- %
% OPTION 4: Minion Pro + Myriad Pro (commercial)
% 
% --------------------------------------------------------------------------- %
%
% Move all *otf fonts into the fonts folder
% 
% load fontspec (not required if you use mathspec, it will load fontspec anyway)
% \usepackage[no-math]{fontspec}
% % see https://tex.stackexchange.com/questions/359704/fontspec-mathspec-clash

% % unicode-math
% % maps math markup into unicode characters provided by OpenType fonts
% % allows using unicode characters intercheangably with markup
% % e.g. \emptyset \not\in X  or ∅ ∉ X
% \usepackage[%
%     warnings-off={mathtools-colon,mathtools-overbracket},
%     math-style=upright]{unicode-math}

% % the Garamond-Math font at this scale is the most similar
% % to the MinionPro font used for the text 
% \setmathfont[Scale=0.95, Numbers={OldStyle,Proportional}]{Garamond-Math}

% % Monospaced font: Inconsolata
% \usepackage{inconsolata}
% \setmonofont[Scale=0.95]{inconsolata}

% declare fonts
% opciones para numeros, kerning, ligatures
% -- ver https://ctan.math.washington.edu/tex-archive/fonts/minionpro/MinionPro.pdf
% -- ver https://ctan.dcc.uchile.cl/macros/unicodetex/latex/fontspec/fontspec.pdf
% 
% -- Numbers (fontspec.pdf page 43)
% -- Upercase, Lowercase, Lining, OldStyle, Proportional, Monospaced, SlashedZero, Arabic
% 
% -- Ligatures (fontspec.pdf page 41)
% -- se pueden definir de a una o varias separadas por comas
% -- Required, Common, Contextual, Historic, Rare, TeX
%

% \setmainfont{MinionPro}[%
%     Path = fonts/,
%     Numbers={OldStyle, Proportional},
%     Kerning=Uppercase,
%     Ligatures={Common},
%     Extension=.otf,
%     UprightFont=*-Regular,
%     ItalicFont=*-It,
%      BoldFont=*-Bold,
%     BoldItalicFont= *-BoldIt,
%     SizeFeatures = {
%         {Size = {5-8.4}, Font = *-Capt, Color = 220c10},
%         {Size = {8.4-13.0}, Font = *-Regular},
%         {Size = {13.1-19.9}, Font = *-Subh},
%         {Size = {20-}, Font = *-Disp},
%     },
% ]

% \setsansfont{MyriadPro}[%
%     Path = fonts/,
%     Extension = .otf,
%     UprightFont = *-Regular,
%     ItalicFont = *-It,
%     BoldFont = *-Bold,
%     BoldItalicFont = *-BoldIt,
% ]