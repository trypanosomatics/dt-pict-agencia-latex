% !TeX root = descripcion-tecnica.tex
% !TeX program = xelatex
% these lines above are important to VSCode + LaTeX workshop 
\title{Descripcion Tecnica PICT 0000} % this is important to Overleaf

% --------------------------------------------------------------------------- %
% LaTeX Preamble
% --------------------------------------------------------------------------- %
\documentclass[spanish,a4paper,11pt]{article}

\def\PICTAuthor{Nombre Apellido}
\def\PICTTitle{Analisis estreptocalifragilistico de ondas supramasquetodas y su impacto en el universo peridomiciliario}
\def\PICTSubject{Descripción Técnica PICT 0000}
\def\PICTKeywords{keyword, keyword, keyword}
\def\PDFCreator{Overleaf, LaTeX, xelatex}
\def\PICTInstitution{Instituto de Investigaciones en Ondas, Unversidad X}
% el archivo de la descripcion tecnica
\def\PICTFile{descripcion-tecnica.pdf}
      % important variables! EDIT!!!
\usepackage[spanish, es-tabla, es-noindentfirst]{babel}	% language support
\usepackage{lipsum}     % Lorem ipsum ... 
\usepackage{url}        % \url
\usepackage{footmisc}   % \footref
\usepackage{longtable}  % to allow footnotes within tables
\usepackage{multirow}   % allow cells spanning multiple columns/rows
\usepackage[dvipsnames,svgnames,table]{xcolor}  % colors
\usepackage{titlesec}   % styling titles of sections
\usepackage{colortbl}   % colors for table cells
\usepackage{graphicx}	% Insert graphics
\usepackage[footnotesize,bf,sc]{caption}	% formatear captions
\usepackage{wrapfig}	% wrap text around figures
\usepackage{soul}		% soul, highlight text in color
\usepackage{tikz}		% required by Gantt Charts
\usepackage{pgfgantt}   % cronograma, timeline
\usepackage{translator} % ?
\usepackage[useregional]{datetime2}   % functions to work with date, time
\usepackage{xstring}	% extract Git commit data
\usepackage{catchfile}	% extract Git commit data
\usepackage{float}      % improved definitions of floating objects (figures, tables)		% import packages
% --------------------------------------------------------------------------- %
% Page settings
% --------------------------------------------------------------------------- %

% needed for page border settings
%\usepackage[top=2cm, bottom=1.8cm,left=3cm,right=3cm]{geometry} 
% Tesis
%\usepackage[top=2cm, bottom=2.5cm, inner=3.5cm, outer=3.5cm]{geometry} 
% PICT 2021
\usepackage[total={6.5in,9in}, left=0.5in, includefoot]{geometry}

% lines and paragraphs
% indentation of first line of new paragraph
%\parindent=0cm 
\parindent=0.7cm 
% separation between paragraphs
\parskip=0.15cm

% try to reduce some blank space
\setlength{\evensidemargin}{-0.1in}
\setlength{\oddsidemargin}{-0.1in}
\setlength{\textwidth}{5.8in}
% \setlength{\topmargin}{-0.5in}
% \setlength{\textheight}{9in}

% try to reduce some blank space
%\setlength{\evensidemargin}{0.0in}
%\setlength{\oddsidemargin}{0.0in}
%\setlength{\textwidth}{6.5in}
%\setlength{\topmargin}{-0.5in}
%\setlength{\textheight}{9in}

\tolerance=600
\widowpenalty=10000
\clubpenalty=10000

\renewcommand{\baselinestretch}{.95}
\normalsize


			% page layout: margins
% --------------------------------------------------------------------------- %
% Git commit data 
% --------------------------------------------------------------------------- %
\CatchFileDef{\headfull}{.git/HEAD.}{}
\StrGobbleRight{\headfull}{1}[\head]
\StrBehind[2]{\head}{/}[\branch]
\IfFileExists{.git/refs/heads/\branch.}{%
    \CatchFileDef{\commit}{.git/refs/heads/\branch.}{}}{%
    \newcommand{\commit}{\dots~(in \emph{packed-refs})}}

			% TeX macros 
% Define light and dark Microsoft blue colours
% used for links in PDF (URLs, citations, etc)
\definecolor{MSBlue}{rgb}{.204,.353,.541}
\definecolor{MSLightBlue}{rgb}{.31,.506,.741}
\definecolor{LinksBlue}{RGB}{30,39,255}
% used for captions
% spacecadet = RGB(21,30,63)
\definecolor{spacecadet}{HTML}{151e3f}
% used for tables
\definecolor{tableShade}{gray}{0.9}
% used in ganttchart
\definecolor{barblue}{RGB}{153,204,254}
\definecolor{groupblue}{RGB}{51,102,254}
\definecolor{linkred}{RGB}{165,0,33}

			% colors used throughout the document
% --------------------------------------------------------------------------- %
% Fonts
% --------------------------------------------------------------------------- %
% if using commercial fonts, place all fonts into their own './fonts' folder 
% .gitignore is already set to avoid uploading fonts to public repos

% --------------------------------------------------------------------------- %
% Option 0: No selection of fonts whatsoever
% --------------------------------------------------------------------------- %
% You will get Computer Modern fonts. 
% --------------------------------------------------------------------------- %

% --------------------------------------------------------------------------- %
% OPTION 1: Stix + Liberation Sans + Inconsolata
% --------------------------------------------------------------------------- %
% https://www.lfe.pt/latex/fonts/typography/2022/11/21/latex-fonts-part1.html
% https://github.com/linoferreira/latex-font-survey/blob/master/stix/lualatex/stix-two.tex
% \usepackage[cm-default]{fontspec}
% \usepackage{microtype}
% \usepackage{mathtools,amssymb}
% \usepackage{unicode-math}
% \usepackage{inconsolata}

% % Serif font: STIX Two (https://www.stixfonts.org)
% \setmainfont[Numbers=OldStyle]{STIX Two Text}
% % Math font: STIX Two Math
% \setmathfont{STIX Two Math}
% % Sans-serif font: Libertinus Sans
% \setsansfont[Numbers=OldStyle]{Libertinus Sans}
% % Monospaced font: Inconsolata (https://www.ctan.org/pkg/inconsolata)
% \setmonofont{inconsolata}
% END OPTION 1
% --------------------------------------------------------------------------- %


% --------------------------------------------------------------------------- %
% OPTION 2: TeX Gyre Pagella + Source Sans Pro + TeX Gyre Cursor
% --------------------------------------------------------------------------- %
% https://www.lfe.pt/latex/fonts/typography/2022/11/21/latex-fonts-part1.html
% https://tug.org/FontCatalogue/texgyrepagella
% https://tug.org/FontCatalogue/texgyrecursor/
%\usepackage{fontspec}
%\usepackage{amsfonts}
% Serif Font: TeX Gyre Pagella
%\usepackage{tgpagella}
%\setmainfont{TeX Gyre Pagella}
% Sans Serif Font: Source Sans Pro
%\usepackage{sourcesanspro}
%\setsansfont{Source Sans Pro}
% Math Font:
% Monospaced Font: 
%\usepackage{tgcursor}
%\setmonofont{TeX Gyre Cursor}
% --------------------------------------------------------------------------- %



% --------------------------------------------------------------------------- %
% OPTION 3: EB Garamond + Source Sans Pro + Garamond Math + New CM Mono
% --------------------------------------------------------------------------- %
% https://www.lfe.pt/latex/fonts/typography/2022/11/21/latex-fonts-part1.html
 \usepackage{fontspec}
% % Serif Font: 
\setmainfont[Numbers=OldStyle]{EB Garamond} 

% % Sans Serif Font: Source Sans Pro
\usepackage{sourcesanspro}
\setsansfont{Source Sans Pro}

% % Math Font
\usepackage[math-style=ISO, bold-style=ISO]{unicode-math}
\setmathfont{Garamond-Math.otf}[StylisticSet={6,7,9}]

% % Monospaced Font
\usepackage{unicode-math}
\setmonofont[ItalicFont=NewCMMono10-Italic.otf,%
    BoldFont=NewCMMono10-Bold.otf,%
    BoldItalicFont=NewCMMono10-BoldOblique.otf,%
    SmallCapsFeatures={Numbers=OldStyle}]{NewCMMono10-Regular.otf}
% --------------------------------------------------------------------------- %


% --------------------------------------------------------------------------- %
% OPTION 4: Libertinus + Inconsolata
% --------------------------------------------------------------------------- %
% https://www.lfe.pt/latex/fonts/typography/2022/11/21/latex-fonts-part1.html
%https://github.com/linoferreira/latex-font-survey/blob/master/libertine/lualatex/libertinus.tex
% \usepackage{fontspec}
% \usepackage{microtype}
% \usepackage{mathtools,amssymb}
% \usepackage{unicode-math}
% \usepackage{xcolor}

% % Serif font: Libertinus Serif (https://github.com/alif-type/libertinus)
% \setmainfont[Numbers=OldStyle]{Libertinus Serif}
% % Math font: Libertinus Math
% \setmathfont[AutoFakeBold]{Libertinus Math}
% % Sans-serif font: Libertinus Sans
% \setsansfont[Numbers=OldStyle]{Libertinus Sans}
% % Monospaced font: Inconsolata (https://www.ctan.org/pkg/inconsolata)
% \fontspec{inconsolata}
% % Display font: Libertinus Serif Display
% \newfontfamily\dispfont{Libertinus Serif Display}
% --------------------------------------------------------------------------- %


% --------------------------------------------------------------------------- %
% OPTION 4: Minion Pro + Myriad Pro (commercial)
% 
% --------------------------------------------------------------------------- %
%
% Move all *otf fonts into the fonts folder
% 
% load fontspec (not required if you use mathspec, it will load fontspec anyway)
% \usepackage[no-math]{fontspec}
% % see https://tex.stackexchange.com/questions/359704/fontspec-mathspec-clash

% % unicode-math
% % maps math markup into unicode characters provided by OpenType fonts
% % allows using unicode characters intercheangably with markup
% % e.g. \emptyset \not\in X  or ∅ ∉ X
% \usepackage[%
%     warnings-off={mathtools-colon,mathtools-overbracket},
%     math-style=upright]{unicode-math}

% % the Garamond-Math font at this scale is the most similar
% % to the MinionPro font used for the text 
% \setmathfont[Scale=0.95, Numbers={OldStyle,Proportional}]{Garamond-Math}

% % Monospaced font: Inconsolata
% \usepackage{inconsolata}
% \setmonofont[Scale=0.95]{inconsolata}

% declare fonts
% opciones para numeros, kerning, ligatures
% -- ver https://ctan.math.washington.edu/tex-archive/fonts/minionpro/MinionPro.pdf
% -- ver https://ctan.dcc.uchile.cl/macros/unicodetex/latex/fontspec/fontspec.pdf
% 
% -- Numbers (fontspec.pdf page 43)
% -- Upercase, Lowercase, Lining, OldStyle, Proportional, Monospaced, SlashedZero, Arabic
% 
% -- Ligatures (fontspec.pdf page 41)
% -- se pueden definir de a una o varias separadas por comas
% -- Required, Common, Contextual, Historic, Rare, TeX
%

% \setmainfont{MinionPro}[%
%     Path = fonts/,
%     Numbers={OldStyle, Proportional},
%     Kerning=Uppercase,
%     Ligatures={Common},
%     Extension=.otf,
%     UprightFont=*-Regular,
%     ItalicFont=*-It,
%      BoldFont=*-Bold,
%     BoldItalicFont= *-BoldIt,
%     SizeFeatures = {
%         {Size = {5-8.4}, Font = *-Capt, Color = 220c10},
%         {Size = {8.4-13.0}, Font = *-Regular},
%         {Size = {13.1-19.9}, Font = *-Subh},
%         {Size = {20-}, Font = *-Disp},
%     },
% ]

% \setsansfont{MyriadPro}[%
%     Path = fonts/,
%     Extension = .otf,
%     UprightFont = *-Regular,
%     ItalicFont = *-It,
%     BoldFont = *-Bold,
%     BoldItalicFont = *-BoldIt,
% ]			% typographic fonts
% Hyphenation
% aca poner ayudas para separar correctamente en silabas palabras dificiles
\hyphenation{iso-pen-ta-no-per-hi-dro-fe-nan-tre-no}

	% como separar en silabas palabras dificiles
% Hyperref for PDFs
\usepackage[pdfborder={0 0 0}]{hyperref}
\hypersetup{
colorlinks = true,
linkcolor = MSLightBlue,
urlcolor = LinksBlue,
citecolor = LinksBlue,
pdfauthor = {\PICTAuthor},
pdftitle = {\PICTTitle},
pdfsubject = {\PICTSubject},
pdfkeywords = {\PICTKeywords},
pdfcreator = {\PDFCreator}
}		% PDF setup
% --------------------------------------------------------------------------- %
% Manejo y estilo de Bibliografia
% --------------------------------------------------------------------------- %
% biblatex - ver 
% https://www.overleaf.com/learn/latex/Biblatex_bibliography_styles
\usepackage[backend=biber,style=nature]{biblatex}
\usepackage{csquotes}
\addbibresource{content/sample.bib}

% reduce font size of bibliography to make 
%\def\bibfont{\footnotesize}
\renewcommand{\bibfont}{\normalfont\small}	

	% setup biblatex, point to .bib file
% --------------------------------------------------------------------------- %
% Captions
% --------------------------------------------------------------------------- %
\captionsetup{
    margin = 5pt,%
    font = {color=spacecadet,small},%
    labelfont = {sc, bf},%
    textfont = it,%
    labelsep = endash,%
    belowskip = 0pt,
}

% --------------------------------------------------------------------------- %
% Sections
% --------------------------------------------------------------------------- %
% requires titlesec package - color of section title text 
\titleformat*{\section}{\Large\bfseries\sffamily\color{MSBlue}}
\titleformat*{\subsection}{\large\bfseries\sffamily\color{MSLightBlue}}
\titleformat*{\subsubsection}{\itshape\sffamily\color{MSLightBlue}}


% --------------------------------------------------------------------------- %
% URLs
% --------------------------------------------------------------------------- %
%% Define a new style for the url package to use a smaller font
%% Taken from http://hex.lspace.org/leo/thesis/tips/url-formatting.html
\makeatletter
\def\url@fernanstyle{\def\UrlFont{\footnotesize\ttfamily}}
\makeatother
\urlstyle{fernan}


			% styles of captions, sections

% --------------------------------------------------------------------------- %
% LaTex Document (content)
% --------------------------------------------------------------------------- %
\begin{document}

% title page
\title{\vspace*{-50pt} \PICTTitle}

\author{\vspace{-50pt} Investigador Responsable: \PICTAuthor}

\date{
  %\vspace{-30pt} 
  \PICTInstitution{}\\[2ex]
  \DTMlangsetup[spanish]{showdayofmonth=false}
  \Today%
}

\maketitle
\tableofcontents



% objetivos generales, especificos + hipotesis - 2 paginas
% --------------------------------------------------------------------------- %
\section{Objetivos generales}\label{sec:objetivos-generales}
% --------------------------------------------------------------------------- %
% max 1 pag
% Objetivos Generales e impacto: Identificar el problema general en 
% estudio, contextualizar el problema a nivel local, identificar que 
% parte del problema se intenta abordar /contribuir con la investigación.
% --------------------------------------------------------------------------- %

\begin{itemize}

  \item Objetivo 1
 
  \item Objetivo 2

  \item Objetivo 3
 
\end{itemize}
%\newpage

% --------------------------------------------------------------------------- %
\section{Objetivos específicos}\label{sec:objetivos-especificos}
% --------------------------------------------------------------------------- %
% max 1 pag
% Identificar los Objetivos específicos relacionados con el problema 
% que se abordará. Describir la hipótesis de trabajo y como se 
% abordará el problema en cuestión a través de la experimentación y estudio.
% --------------------------------------------------------------------------- %


\subsection{Primer objetivo general}

 
  \begin{enumerate}
    \item Aim 1 
    \item Aim 2
    \item Aim 3
  \end{enumerate}

\subsection{Otro objetivo geneeral}

  \begin{enumerate}
    \setcounter{enumi}{3}

    \item Aim 4
    \item Aim 5
    \item Aim 6

  \end{enumerate}



% --------------------------------------------------------------------------- %
% Hipotesis de trabajo
% --------------------------------------------------------------------------- %
% Describir la hipótesis de trabajo y cómo se abordará el problema en 
% cuestión a través de la experimentación y el estudio.
% --------------------------------------------------------------------------- %

\section{Hipótesis de trabajo}\label{sec:hipotesis}

Nuestra hipótesis de trabajo es ... 

% relevancia - 3 paginas
% --------------------------------------------------------------------------- %
% Relevancia 
% --------------------------------------------------------------------------- %
% max 3 pags
% Desarrollar la importancia e impacto a nivel local, general y para
% la especialidad del problema, los objetivos y el conocimiento que se
% generará. Describir antecedentes, avances y el estado del arte. 
% Búsqueda bibliográfica actualizada.
% --------------------------------------------------------------------------- %
\section{Relevancia e Impacto}\label{sec:relevancia}

Let's cite something so we can have references! The Einstein's journal paper
\cite{einstein} and the Dirac's book \cite{dirac} are physics related items.
Next, \textit{The \LaTeX\ Companion} book \cite{latexcompanion}, the Donald
Knuth's website \cite{knuthwebsite}, \textit{The Comprehensive Tex Archive
Network} (CTAN) \cite{ctan} are \LaTeX\ related items; but the others Donald
Knuth's items \cite{knuth-fa} are dedicated to programming.


% resultados preliminares - 3 paginas
% --------------------------------------------------------------------------- %
% Resultados y datos preliminares
% --------------------------------------------------------------------------- %
% max 3 pag
% Describir con suficiente detalle los resultados ya obtenidos por 
% el grupo, sean publicados o no, que indican la capacidad técnica 
% del grupo y la dedicación previa del grupo para el estudio 
% propuesto.
% --------------------------------------------------------------------------- %
\section{Resultados preliminares y aportes del grupo}\label{sec:aportes}




% proyecto (investigacion + metodos) - 9 paginas
% --------------------------------------------------------------------------- %
% Proyecto - Diseño experimental y metodos 
% --------------------------------------------------------------------------- %
% max 9 pags
% Se deberá organizar el estudio propuesto en secciones mayores
% correspondientes a los objetivos específicos y secciones menores
% correspondientes a experimentos específicos para explicar:
%  1. La base racional de cada experimento o estudio propuesto.
%  2. Como se llevara a cabo el experimento o estudio.
%  3. Que controles se usaran - en caso de ser necesarios - y porque.
%  4. Que técnicas especificas se utilizaran discutiendo aspectos
%     más críticos o modificaciones de manipulaciones habituales: 
%     Respecto a las técnicas y tecnologías empleadas (los métodos) 
%     si son parte del patrimonio del grupo y han sido descriptas en
%     publicaciones propias o en los datos preliminares - no deberán 
%     detallarse y solo deberá citarse la fuente. Explicar si se 
%     recibirá apoyo técnico de colaboradores.
%  5. Como se interpretaran los datos a la luz de lo que se quiere
%     estudiar y como se contrastará con la hipótesis de trabajo.
%  6. Tratar de evaluar los potenciales problemas y limitaciones de
%     la metodología y técnicas propuestas y en lo posible proponer
%     alternativas.
% --------------------------------------------------------------------------- %
\section{Diseño experimental y métodos}\label{sec:experimental} 



% cronograma - 1 pagina
% -------------------------------------------------------------------
\section{Cronograma de trabajo} 
% -------------------------------------------------------------------
% max 1 pag Se presentará una tabla de doble entrada con las tareas
% desagregadas y los tiempos estimados que consumirán.
% -------------------------------------------------------------------
\makeatletter
\setlength{\@fptop}{0pt}
\makeatother

\begin{figure}[H]
	% --------------------------------------------------------------------------- %
% CRONOGRAMA as Gannt Chart
% --------------------------------------------------------------------------- %
\setganttlinklabel{s-s}{START-TO-START}
\setganttlinklabel{f-s}{FINISH-TO-START}
\setganttlinklabel{f-f}{FINISH-TO-FINISH}
\sffamily

% scale down the whole chart to fit the page 
\scalebox{0.85}{
\begin{ganttchart}[
    canvas/.append style={fill=none, draw=black!5, line width=.75pt},
    hgrid style/.style={draw=black!5, line width=.75pt},
    vgrid={*1{draw=black!5, line width=.75pt}},
    today=2,
    today rule/.style={
      draw=black!64,
      dash pattern=on 3.5pt off 4.5pt,
      line width=1.5pt
    },
    today label font=\small\bfseries,
    today label=INICIO PICT,
    title/.style={draw=none, fill=none},
    title label font=\bfseries\footnotesize,
    title label node/.append style={below=7pt},
    include title in canvas=false,
    bar label font=\mdseries\small\color{black!70},
    bar label node/.append style={left=2cm},
    bar/.append style={draw=none, fill=black!63},
    bar incomplete/.append style={fill=barblue},
    bar progress label font=\mdseries\footnotesize\color{black!70},
    group incomplete/.append style={fill=groupblue},
    group left shift=0,
    group right shift=0,
    group height=.5,
    group peaks tip position=0,
    group label node/.append style={left=.2cm},
    group progress label font=\bfseries\small,
    link/.style={-latex, line width=1.5pt, linkred},
    link label font=\scriptsize\bfseries,
    link label node/.append style={below left=-2pt and 0pt}
  ]{1}{10}
  \gantttitle[
    title label node/.append style={below left=7pt and -3pt}
  ]{SEMESTRES:\quad1}{1}
  \gantttitlelist{2,...,10}{1} \\
  \ganttgroup[progress=35]{Objetivo 1}{1}{6} \\
  \ganttbar[progress=60]{\textbf{\ref{sec:aim1}} Aim 1}{1}{4} \\
  \ganttbar[progress=0]{\textbf{\ref{sec:aim2}} Aim 2}{2}{4} \\
  \ganttbar[progress=0]{\textbf{\ref{sec:aim3}} Aim 3}{4}{5} \\
  \ganttbar[progress=0]{Publish results}{3}{6} \\[grid]
  \ganttgroup[progress=20]{Objetivo 2}{1}{10} \\
  \ganttbar[progress=40]{\textbf{\ref{sec:aim4}} Aim 4}{1}{5} \\
  \ganttbar[progress=20]{\textbf{\ref{sec:aim5}} Aim 5}{2}{7} \\
  \ganttbar[progress=0]{\textbf{\ref{sec:aim6}}  Aim 6}{5}{10}\\
  \ganttbar[progress=0]{Publish results}{7}{10}
  %\ganttlink[link type=s-s]{WBS1A}{WBS1B}
  %\ganttlink[link type=f-s]{WBS1B}{WBS1C}
  %\ganttlink[
  %  link type=f-f,
  %  link label node/.append style=left
  %]{WBS1C}{WBS1D}
\end{ganttchart}
}


	\caption{{\bfseries{Cronograma de Objetivos y Tareas}}. Este proyecto es
	continuación del proyecto PICT-0000-0000 (Agencia I+D+i). En el
	cronograma se muestran tareas ya iniciadas que forman parte de estos
	proyectos. El diagrama muestra las tareas desagregadas y los tiempos que
	consumirán cada una. La fecha de inicio de este subsidio se marca como
	referencia tentativa (línea punteada). Los semestres 1 y 2 corresponden al
	período anterior al inicio de este proyecto solicitado. Los objetivos y
	tareas completados se muestran en distintas escalas de negro/gris, aquellos
	en progreso en azul.}\label{fig:cronograma}

\end{figure}


% --------------------------------------------------------------------------- %
% Alternativa 
% --------------------------------------------------------------------------- %
% tomar un screenshot de un cronograma, ej hecho en una planilla
% e incrustarlo acá
%\begin{figure}[!h]
%  \includegraphics{figures/cronograma}
%  \caption{Cronograma.}\label{fig:cronograma}
%\end{figure}

 

% bibliografia - max 2 paginas
% --------------------------------------------------------------------------- %
% Bibliografia
% --------------------------------------------------------------------------- %
% retocar spacing + baselinestretching 
% para ajustar espacio que ocupa la bibliografia 
%\singlespacing (\usepackage{setspace}?)
\renewcommand{\baselinestretch}{.7}

\addcontentsline{toc}{section}{Referencias}
%\bibliographystyle{naturemag}
%\bibliography{content/zotero.bib}
\printbibliography%

% final - filestats
% --------------------------------------------------------------------------- %
% File stats
% --------------------------------------------------------------------------- %

\vspace{24pt}
\ttfamily
\noindent Archivo: \jobname.pdf \\
Tamaño del archivo: \the\numexpr(\filesize{\PICTFile}/1024)K \\
Revisión Git: \texttt{\commit}(\texttt{\branch})\\
Longitud del archivo: \the\numexpr\value{page}~páginas \\

%Descripción Técnica:
%\the\numexpr\getpagerefnumber{LastPage}-\value{page}\relax~páginas (excluyendo
%carátula + índice) \\




\end{document}
